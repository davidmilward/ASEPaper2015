
\section{Related Work}

\subsection{ISO/IEC 11179}

The work described in this paper has evolved from the CancerGrid
project~\cite{davi14}, where an ISO11179-compliant metadata registry
was developed for curation of semantic metadata and model-driven
generation of trial-specific software~\cite{davi12, Abler2011}.  The
approach to generating forms in the CancerGrid project has been
generalised significantly with the introduction of a data modeling
language and a broader notion of semantic linking. 

Another effort to develop an implementation of ISO11179 is found in
the US caBIG initiative~\cite{kunz09}; however, their caCORE software
development kit~\cite{koma08} applies model-driven development only to
generate web service stubs, requiring developers to create application
logic by hand, whereas our technique integrates with existing clinical
Electronic Data Capture tools and workflows, such as
OpenClinica~\cite{oc}.

\subsection{Ontology modelling}

%%%% ontological representations of ISO11179 %%%%%%%%

Several efforts have addressed ontological representations of ISO11179 
for enabling data integration across metadata
registries (MDRs). Sinaci and Erturkmen~\cite{Sinaci2013784} describe a
federated semantic metadata registry framework where Common Data
Elements (CDEs) are exposed as Linked Open Data resources. CDEs are
described in the Resource Description Framework (RDF), and can be queried and interlinked with CDEs in other
registries using the W3C Simple Knowledge Organization System (SKOS). 
An ISO11179 ontology has been defined as part of
the framework, and the Semantic MDR has been implemented using the Jena
framework. 
Jeong \textit{et~al.}~\cite{pmid25405066} present the Clinical
Data Element Ontology (CDEO) for unified indexing and retrieval of
elements across MDRs; they organise and
represent CDEO concepts using SKOS. 
Tao \textit{et~al.}~\cite{pmid22211181} present
case studies in representing HL7 Detailed Clinical Models (DCMs) and
the ISO11179 model in the Web Ontology Language (OWL);
a combination of UML diagrams and Excel
spreadsheets were used to extract the metamodels for fourteen HL7 DCM
constructs. A critical limitation of this approach is that the
transformation from metamodels to their ontological representation in
OWL is based on a manual encoding. 
Leroux \textit{et~al.}~\cite{lero12} use existing
ontologies to enrich OpenClinica forms; our Model Catalogue
technique can integrate these ontologies to capture compliant data in
a similar fashion, but does so by using an ISO11179 meta data registry
and model driven methodology.

%%%% Ontology repositories %%%%%%%%%

Ontology repositories can be considered closely analogous to model
catalogues, they provide the infrastructure for storing, interlinking,
querying, versioning and visualising ontologies. Relationships
capturing the alignments and mappings between ontologies are also
captured, allowing easy navigability. Linked Open
Vocabularies~\cite{LOV} provides a service for discovering
vocabularies and ontologies published following the principles of
linked data. Besides the above features, it provides documentation
about ontologies automatically harvested from their structure and
identifies dependencies. Apache Stanbol~\cite{Stanbol} provides a set
of reusable components for semantic content management. The Apache
Stanbol Ontology Manager provides a controlled environment for
managing ontologies and ontology networks. 
%Main components of the
%manager include (a) Ontonet, a Java API for the construction and
%management of ontology networks from the ontology knowledge base. (b)
%Registry, an RDF resource that provides descriptions of ontology
%libraries in the knowledge base. The registry also provides a
%management API for administrators to pre-configure sets of ontologies
%loaded in the ontology libraries. KAON2~\cite{Kaon2} provides an API
%infrastructure for managing OWL-DL, SWRL and F-Logic
%ontologies. Access to the ontologies is provided via a single stand
%alone server. OWL DL Inferencing and SPARQL based querying are
%supported.


%%%%%%%%%%%%%%%%%%%%%%%%%%%%%%%%%

\subsection{Data warehousing}

In data warehousing~\cite{kim02}, metadata is modelled in order to copy
information from business systems into a centralised `data warehouse',
for decision support and to analyse business performance. Data
warehouse models can be arranged in normalised form, following the
relational approach~\cite{inm92}, or `dimensional' form~\cite{kim02}
as quantifiable \emph{facts} and \emph{dimensions} which denote the
context of facts. The Common Warehouse Metamodel (CWM)~\cite{poole03}
from the Object Management Group is a UML-based framework to enable
data warehousing in practice. Data warehousing is focused on
write-once models, and for working with rigidly structured data; this
is in contrast to the more general approach taken in ISO11179, where
models are expected to evolve and change over time. 
The core metamodel of the CWM standard overlaps with the 
\emph{concept} and \emph{value} elements of the ISO11179 meta model.

\subsection{Model-driven engineering for e-Health}

Several examples of model-driven engineering for e-Health software are
reported in the literature~\cite{dav14,ragh08,blob07,kham08,schl15}.
Payne~\cite{pay12} formalises the typical pattern followed in these
methods: a multi-phase approach, where data modelling is a separate
phase from stakeholder engagement and data integration. This approach
is taken Khambati \textit{et~al.}~\cite{kham08}, where an
Eclipse-based tool is used to develop domain-specific languages to
model and generate tools for mobile health-tracking applications. The
advantage of the approach is the ability it provides for clinicians to
modify the model of the study, which is specified in the DSL, and
automatically regenerate the application from the model. A similar
approach is taken in the \emph{True Colours} system~\cite{dav14},
using the Booster model-driven toolkit to derive a patient
self-monitoring application for mental health. The Booster approach
demonstrates the lesson that data tends to be managed better within a
model-driven process, leading to higher quality and more reusable
assets.  Schlieter \textit{et~al.}~\cite{schl15} record their
experience gained from using model-driven engineering to implement an
application for path-based stroke care; amongst the lessons learned,
they recommend using existing ontological models where possible, and
being prepared to reconcile a heterogeneity of models from the various
stakeholders under a common metamodel.  In contrast to these systems,
our metadata-oriented approach supports the creation of applications
that can interoperate with existing data, standards and
systems. Rather than simply using MDE to develop stand-alone systems,
MDE processes are used in the management of clinical trials metadata
from which software is derived.

In the Model Driven Health Tools (MDHT)~\cite{MDHT} project, the HL7
Clinical Document Architecture (CDA) standard~\cite{doli06} for
managing patient records is implemented using Eclipse UML
tools~\cite{EUML}. The benefits of applying MDE
are clear: modelling tools are used to model the CDA standards, and
interoperable implementations of the standard are automatically
derived from the models. In principle, this is similar to 
our Model Catalogue approach, where the CDA metadata can be represented and
implementations derived. However, MDHT supports only the CDA standard,
whereas the Model Catalogue can interoperate with any metadata
standard. The CDA standards are large and complex: 
Scott and Worden~\cite{sco12} advocate a
model-driven approach to simplify the HL7 CDA,
supported by three case studies: the NHS England `Interoperability
Toolkit', simplification of US CDA documents, and the Common
Assessment Framework project for health and care providers in
England. The Model Catalogue supports
similar simplifications, where large and complex metadata schemes are
simplified by mapping only relevant metadata in the generated
artefacts.

\subsection{Electronic data capture}

A range of tools are available for clinical Electronic Data Capture
(EDC), including Catalyst Web Tools~\cite{catalyst},
OpenClinica~\cite{oc}, REDCap~\cite{harr09}, LabKey~\cite{labk} and
Caisis~\cite{cais}. Franklin \textit{et~al.}~\cite{fran11} present
a two-year case-study comparing EDC tools, 
and Leroux \textit{et~al.}~\cite{lero11} report on
a further study comparing Clinical Trials Management Systems. 
We use the Model Catalogue tool to generate
case report forms for OpenClinica, but in principle any EDC tool could be
supported. 

The state of the art in EDC is represented by the Research Electronic Data
Capture (REDCap)~\cite{harr09} web-based application, which
supports metadata capture for research studies, providing an online
interface for data entry, audit trails, export to common statistical
packages and data import from external sources. Like the Model
Catalogue, the REDCap system focuses on the clinical
metadata. However, REDCap and similar EDC tools are typically 
insular systems, importing any data into a centralised data silo;
in contrast, the Model
Catalogue aims to provide a platform to support and integrate 
existing data stores
and systems within the clinical environment.

There is also a distinction in the level of expertise expected to operate
the tools. The metadata management --- creation, revision, sharing~--- in
EDC is typically considered a IT-specialist task~\cite{harr09,fran11}, requiring
experts to initialise the metadata separately for each study. With the Model
Catalogue, clinical domain specialists have the ability to adapt and modify
metadata as needed, and treat models as the central
artefacts. Effectively, the Model Catalogue follows the same metadata
workflow as REDCap, but without the need for modelling experts to
develop, adapt or share the metadata models. 

%REDCap, unlike the
%Model Catalogue, is not open source so users cannot freely modify the
%functionality of the system. The Model Catalogue, can be modified to
%add features and support to maximise the utility of the
%models. Because the Model Catalogue works at the meta-data level,
%users can share similar models between organisations and derive new
%software artefacts from user created metadata models.

