
\section{Related Work}
State-of-the-art is the research electronic data capture (REDCap)~\cite{harr09} web-based application to support translational research. The tool supports data capture for research studies providing an online interface for data entry, audit trails, export to common statistical packages and data import from external sources. Like the Model Catalogue, the REDCap system focuses on the clinical metadata. However, REDCap differs from the current Model Catalogue in several important aspects. Firstly, REDCap is a insular and centralised system, any data must be imported into and stored within the REDCap database, so acts as a silo for data. The Model Catalogue aims to provide a platform to support existing data stores and systems within the clinical environment. The meta-data management (creation, revision, sharing) in REDCap is considered a specialist task, requiring experts to initialise a REDCap installation per-study. In the Model Catalogue, clinical users have ability to adapt and modify the metadata as needed, and treat models as the central artefact of study, in a model-driven way. Effectively, the Model Catalogue follows the same metadata workflow as REDCap, but without the need for modelling experts to develop, adapt or to share the meta data models. REDCap, unlike the Model Catalogue, is not open source so users cannot freely modify the functionality of the system. The Model Catalogue, can be modified to add features and support to maximise the utility of the models. Because the Model Catalogue works at the meta-data level, users can share similar models between organisations and derive software artefacts directly from user created metadata models.


Data warehousing~\cite{kim02} provides a framework for reporting and analysis tasks across disparate enterprise data systems, for decision support. In data warehousing meta data is modelled to extract information from business systems into a centralised data warehouse. The warehouse is used to create reports and analyse business performance. Data warehouse models can be arranged in normalised form, following the relational approach~\cite{inm92}, or `dimensional’ form~\cite{kim02} as quantifiable \emph{facts} and \emph{dimensions} which denote the context of facts. Data warehousing relies on fixed data models and structures, which is in contrast to the more general approach taken in ISO11179, where models are expected to change over time. 

The Common Warehouse Metamodel (CWM)~\cite{poole03} from the Object Management Group is a UML based framework to enable data warehousing in practice. However, as with data warehousing, CWM is focused on write-once models, and for working with rigidly structured data. The models and data warehouse structure in CWM are not intended to change after creation and data is copied into the data warehouse from separate systems. The standard consists of 22 parts and the core meta model has overlap with the the \emph{concept} and \emph{value} elements of the ISO11179 meta model.




%%%%  ontological representations of ISO11179 %%%%%%%%

Several efforts have addressed the representation of ISO11179 as
ontological models for enabling data integration across metadata
registries (MDRs). In \cite{Sinaci2013784} the authors describe a
federated semantic metadata registry framework where CDE (Common Data
Elements) are exposed as LOD resources. CDEs are described in RDF, can
be queried and interlinked with CDEs in other registries using
SKOS. An ISO1179 ontology has been defined as part of the framework
and the Semantic MDR has been implemented using the Jena framework. In
\cite{pmid25405066} the authors present the Clinical Data Element
Ontology (CDEO) for unified indexing and retrieval of elements across
MDRs. Concepts in CDEO have been organised and represented using
SKOS. In \cite{pmid22211181} the authors present case studies for
representing HL7 Detailed Clinical Models (DCMs) and the ISO11179
model in OWL. A combination of UML diagrams and Excel spreadsheets
were used to extract the metamodels for 14 HL7 DCM constructs. A
critical limitation of this approach is that the transformation from
metamodels to their ontological representation in OWL is based on a
manual encoding.

%%%% Ontology repositories %%%%%%%%%

Ontology repositories can be considered analogous to model catalogues
 they provide the infrastructure for storing, interlinking, querying and visualising 
