\documentclass[conference]{IEEEtran}

\ifCLASSINFOpdf
\DeclareGraphicsExtensions{.pdf,.png,.jpg}
\usepackage[pdftex]{graphicx, hyperref}
\usepackage{listings}
\fi

\usepackage{zed} % version 4 included in directory

\newtheorem{definition}{Definition}

\begin{document} 

\title{Model Driven Engineering for \\ Large-Scale Data Integration}

\vskip 4mm 

\author{%
  \IEEEauthorblockN{Jim Davies, Jeremy Gibbons, David M.~Milward,
    Seyyed Shah, Monika Solanki and James Welch%
  }
  % 
  \IEEEauthorblockA{%
    Department of Computer Science, University of Oxford \\
    Email: \texttt{Firstname.Lastname@cs.ox.ac.uk}
  }
}

\ifpdf
\graphicspath{{ASEFigs/}}
\fi

\maketitle

\begin{abstract}
  Large-scale data integration: combining data from heterogeneous
  sources and providing users with a unified view.  More than that in
  practice: it's not a single unified view, but one for each different
  class of user, and for each of the usages that they have in mind.
  Relies upon achieving semantic interoperability: in terms of the
  data---it would be good to have some data that is actually
  compatible---but also in terms of the metadata---if your data isn't
  compatible with mine, fine, that's just the way it is, but I need to
  know.  We are talking big data.  But our definition of big data is:
  need more automation that we have, because it's too big to do this
  manually, with the tools that we are using at present.  

  Various aspects of the management of the data---specification,
  collection, integration, presentation, application, and feedback to
  specification again---are amenable to automation if we have a
  description of the data that our tools can work with.  Sure, we want
  metadata, but we want descriptive metadata.  We want data models.  

  This paper reports upon the development and application of
  supporting tools for automation of some of these aspects, assisting
  domain experts and software engineers in achieving the kind of
  large-scale data integration needed for effective translational
  research.  It begins with an explanation of the modelling framework
  required.  It describes the design and functionality of the tools,
  based around the 'model catalogue'.  It reports upon the experience
  of applying the tools in providing national support for large-scale
  clinical research.  It sets out some lessons learned. 
\end{abstract}

\vskip 14mm

\noindent

\section{Intoduction}

The UK NIHR Health Informatics Collaberative is backing a cross-Trust Programme across 5 key NHS Hospital trusts in the UK in order to set up a flexible and responsive governance framework, whereby research outcomes can rapidly be exploited by the NHS community. The work is currently limited to 5 clinical areas, but is expected in time to be extended. One of the aims of the programme is to develop tools and services for research, so that researchers can clinicians can have access to a wider cross Biomedical Research Centre (BRC) dataset. The programme has been working on developing a federated metadata registry, based on ISO11179\cite{ISO11179}, for use as a basis for enabling interoperability primarily for research data from clinical trials but also with a view to integrating this capabiliy with Electronic Patient Records (EPR) data within the trusts.

\subsection{Background}


\subsection{ISO11179}


\subsection{Related Work}


\subsection{Objectives}
\section{Data Models and Metadata}

\subsection{Models Catalogue}

A software \emph{model} is an abstraction of a software program, which
will be instantiated and run using \emph{real data}, software
modelling is becoming important for a number of reasons, not least
that it enables developers the opportunity to \emph{re-use} code. The
Unified Modelling Language UML \cite{UML} or Ecore \cite{ECORE} ( a
subset of UML) are in widespread use for defining and representing
software \emph{models}. UML and Ecore are defined at a level which is
\emph{more abstract} than the model itself, and one which we term the
\emph{meta-modelling} layer.

Thus there are two things we need to identify and reference in
building the toolkit, firstly does this \emph{concept} \textbf{mean}
the same as that \emph{concept}, and secondly does this \emph{concept}
have the same \emph{representation} as that concept. The way in which
we can match up concepts in software engineering is by associated them
with classes of software objects, and then defining methods which can
reference them.

\subsection{Model Driven Software Engineering (MDSE)}

Clinical research data is stored in many different formats, and data
which may needed for a study may well originate in many different
places. As well as the precise semantics of the data being difficult
to capture, so to is the syntax. One set of forms may have been
captured using Excel, another using a web-based service such as Survey
Monkey, and yet another is stored in a relational database. The
majority of data comes either from a patient answering a question and
recording the answer on some kind of electronic form representation or
from a clinician answering a question and recording the answer. Many
questions will be similar, for instance \emph{what is the patient's
  weight at a particular time}, but they may have different conditions
attached \emph{after a meal, before taking the medicine}. The problem
when dealing with datasets with such research results is that the
structure of the data varies. Collecting data together for the
purposes of running federated queries across the data is largely a
structural rather than a behavioural problem. MDSE is a branch of
computer science which treats the \emph{model} as the primary object
of study. It also is able to transform data from it's stored
structure, with transformation languages such as ATL ~\cite{ATL}, XSLT
~\cite{XSLT} and QVT ~\cite{QVT} data can be transformed from one
format to another format for use in different systems. MDSE provides
the theoretical background which enables these transformations to be
designed and used in code generation toolkits and in
semi-automatically generated software toolkits.

\subsubsection{Metadata and ISO11179}
ISO11179 is the international standard relating to metadata and in particular metadata registries. The ideas and principles documented in this standard have informed the development of the models catalogue from the early prototypes built as part of the Cancer Grid project \cite{crichton2009metadata}, through to the latest versions of the models catalogue.

\begin{figure}[here]
	\includegraphics[width=0.48\textwidth,natwidth=610,natheight=642]{BasicISO}
	\caption{Core model for ISO11179 Metadata Registry} 
	\label{fig:basicMDR}
\end{figure}

The ISO11179 standard uses the notion of a \emph{data element concept}, \emph{a data element}, \emph{a value domain}, and a \emph{conceptual domain}. The standard currently confines itself to the detailed level of concepts and data elements and has no notion of collections of data elements or data element concepts, but instead attaches two attributes: an \emph{object class} and a \emph{property} to each data element concept and these attributes allow the data element concept's to be aggregated or classified. This core model of the ISO11179 is illustrated in figure \ref{fig:basicMDR}. The data element concept and conceptual domain entities belong to the \emph{conceptual level}  whilst the common data element and value domain both belong to the \emph{representational level}.


For example an Integer data-type in a programming language may be used to represent inches in a measurement program, it may also be used to count vehicles in a logistics application.  A data element is said to be comprised of a data element concept(DEC) which is its meaning and a value domain(VD) which is its representation.

\begin{table}[h]
	\begin{tabular}{ p{1.8cm} p{2.8cm}  p{3.0cm}  }  % centered columns (2 columns)
		\hline
		Entity & ISO Definition & ISO11179 Implementation Guidelines  \\ 
		\hline
		Data Element Concept(DEC) & An idea that can be represented in the form of a data element, described independently of any particular representation. & A concept that can be represented in the form of a Data Element, described independently of any particular representation.\\
		Common Data Element(CDE) & A unit of data for which the definition, identification, representation, and permissible values are specified by means of a set of attributes. & A unit of data for which the definition, identification, representation and Permissible Values are specified by means of a set of attributes. \\
		Value Domain (VD) & The description of a value meaning. & A description of a Value Meaning. \\
		Conceptual Domain (CD) & A set of valid value meanings, which may be enumerated or expressed via a description.& A set of valid Value Meanings.\\
		\hline
	\end{tabular}
\end{table}

The table lists the definitions of the key entities used in the standard
Conceptual domains comprise sets of value domains, they provide a collection mechanism of \emph{Value Meanings} which provide representation for a particular data element concepts. Data Element Concepts can then be grouped according to their Object Class, or Property, however whilst this works for a system that is focussed entirely on the metadata units, any working software system will need to group the data elements in a structure that is easily transformed into the components such as Classes and Entities that are used in most information systems.   



\subsection{Data Modelling and Meta-Modelling}
In our work with clinical trials the metadata registry standard on its own is insufficient to answer our principle use cases, which were to share data models, encourage re-use of data elements where possible, develop new data models, and use those data-models in the generation of software artefacts in particular forms for clinical trials.

Working with the ideas captured in the standard and applying them to our use cases we have developed a \emph{domain specific language for meta-modelling} which internally we are calling LEMMA; it is essentially a domain specific language which was developed for handling of clinical research data, but which could possibly be used in other domains. As a computer language it is  weak, however the main problem we are using it for is to restructure clinical \emph{data-models} and for the task in hand it has proved capable.

The key concept behind the language is that of a datamodel, a datamodel is a model composed of dataclasses and dataelements which holds structured data in a particular subject area. The \emph{models catalogue} toolkit is essentially a registry for these \emph{datamodels}, each one being registered with a version number. The datamodel maintains the structure and relationships between a set of dataclasses and dataelements, each of which essentially represent a \emph{data concept}. Dataclasses are in essence very similar to \emph{entities} in entity-relational diagrams, and to \emph{classes} in object-orientated programming; they are simply data-structures. They can be composed from other dataclasses, or from dataelements. Dataelements are atomic entities which can represent a concept or a component of a concept, they can exist in a datamodel independently or as a compositional attribute inside a dataclass. We have a built a version of the language using the Eclipse Modelling Framework (EMF ~\cite{EMF}), basing the various entities on Ecore~\cite{ECORE} classes. A simplified overview model, without attributes and methods, showing the Ecore model for this DSL is shown in Figure \ref{fig:mcSimplifiedOverview}.
%%------------REDO-----------------------%%
\begin{figure}[here]
	\includegraphics[width=0.5\textwidth,natwidth=610,natheight=642]{MetaModel}
	\caption{Overview of Metamodel} 
	\label{fig:mcSimplifiedOverview}
\end{figure}

\subsubsection{DataModel}
A datamodel is a grouping or containment entity which groups a set of \emph{DataClasses} together. DataModels can be thought of as datasets, or even database schemas, very often in the medical domain they are defined either by XML Schema definition files, or by equivalent schemas written in Excel. 
DataModels are collections of \emph{ConceptElements} which in turn can be either \emph{Data Elements} or \emph{Classes}. There is no real notion of composition or multiplicity, a instance of a DataModel can contain an instance of a Data Element or not as required by the instance.  DataModels are named, have a description and have a version identity.
\subsubsection{DataClass}
A DataClass is a grouping or collection of \emph{attributes} which can be data elements or classes, the attributes are currently \emph{mandatory}, so that DataClass with 5 attributes must have those 5 attributes instantiated in an instance for it to be considered of that DataClass. A DataClass in our meta-modelling language is so named to differentiate it from the term \emph{Class} as used in object oriented programming languages, the main difference being that it captures the structural rather than behavioural aspects of a class.  DataClasses represent \emph{Concepts}, and can be \emph{Generalized} into a hierarchy, giving some of the benefits of inheritance to the language. In essence it enables users to take dataclasses from one datamodel, build a sub-class with all the previous data points and then add to it with new dataclasses and dataelement.
\subsubsection{Data Elements} 
Data Elements can also represent \emph{Concepts} and are by their nature \emph{atomic}.  Each data element is related to a value domain on a one-to-one basis, and the relationship is a two-way relationship.
\subsubsection{Value Domain}
A Value Domain is the domain in which the data element is represented, it can consist of one or more \emph{ValueSpecs}. In addition a valuedomain can have zero or more \emph{Measurement Units}, which are descriptive tags which refer to a measurement unit such as kilometers per hour, imperial pounds, or meters. 
\subsubsection{ValueSpec}
ValueSpecs are intermediate entities to describe \emph{how} the data will be represented. There are 3 ValueSpecs, the first is a \emph{datatype}, the second is an \emph{enumeration} of a datatype, and the last is a \emph{rule}. A valuedomain will have at least one ValueSpec and at most 3 ValueSpecs which must be of different kinds.
The main reason for creating a language for handling the data structures in this manner is not only so that models can be curated, versioned and maintained easily, but that the datamodels produced can then be transformed automatically for use in other heterogeneous systems. To illustrate this figure ~\ref{fig:mofLayers} shows the LEMMA meta-model within a 4 tier abstraction diagram. The layers within this diagram are based around idea put forward by the  OMG(~\cite{OMG}), specifically model driven architecture (MDA ~\cite{MDA}). The MDA defines \emph{n} abstraction layers for modelling, although 4-layers are normally used.
\begin{figure}[here] 
	\includegraphics[width=0.5\textwidth,natwidth=610,natheight=642]{MOFLayers}
	\caption{ Datamodel in Excel Format} 
	\label{fig:mofLayers}
\end{figure}
The M0 layer is \emph{real-world} level, the level at which real world objects exist, people, cars, programs, etc. The M1 layer is level at which programs, such as a java program, or a java class definition exist in their static runtime form. The instantiation of a class occurs at the level below the class, so a java object runs in level M0. In defining datasets or datamodels we are dealing in datamodels which are instantiated at the M1 level, the \emph{Cancer and Outcomes Services Dataset} exists at this level, although their may be many conforming instances running at the M0 level. The datamodel is defined in out meta-modelling language, which exists at the M2 or meta-modelling level, and in turn conforms to the more abstract specification known as ECore.

An instance of a datamodel can be written in this DSL in the same way as java code, and figure \ref{fig:excelCOSD} shows how this code looks in the eclipse development toolkit. In this diagram the excel representation of the \emph{Cancer and Outcomes Services Dataset} shown in figure \ref{fig:excelCOSD} has been transformed into the meta-model.

\begin{figure}[here]
	\includegraphics[width=0.5\textwidth,natwidth=610,natheight=642]{COSDExcelS}
	\caption{ Datamodel in Excel Format} 
	\label{fig:excelCOSD}
\end{figure}
The DSL is used to represent a model for Cancer data (part of the Cancer Outcomes and Services Dataset ~\cite{COSD}) in figure \ref{fig:elmcosd}, and is being used within the eclipse toolkit. The DSL can be used to transfer datasets between instances of the models catalogue, although other formats such as JSON and XML are also available.

\begin{figure}[here]
	\includegraphics[width=0.5\textwidth,natwidth=610,natheight=642]{MCCOSDModelS}
	\caption{ Datamodel in Eclipse environment} 
	\label{fig:elmcosd}
\end{figure}

\subsection{Forms and Automatic Software Generation}

By defining the LEMMA metamodel and a similar and dependent Forms metamodel, both at the M" level, and both \emph{Platform Independent} we are able to relate curated data elements to forms elements, and by selecting a group of dataelements and dataclasses automatically generate a set of related forms. This can be done in several ways, the most used way is to generate a platform independent model and then use that model to generate an Excel template. 



\section{Implementation}

The Models Catalogue, like many software developments, evolved from a mix of requirements on a number of different projects.  Initially a metadata registry was built using an XML database, however problems were encountered with scalability.  Most of the language and metamodel developments were carried out using XText and the Eclipse Modelling Framework, resulting in a usable java code base, however usability requirements neccessitated the refactoring of this software using a stack consisting of Grails and Angular JS. 

Grails is built on the Spring framework, which is not only proven to be very robust and scalable, but is also relatively easy to implement and so enables quick agile development cycles. Previous implementations using Java/Spring and Java/Roo have proved very time-consuming to experiment with, whereas Grails has proven to be more flexible and easier to experiment with.  Domain specific languages (DSL’s) can be  built on this framework, and this capability offered scope to build a DSL based on the LEM DSL specification and meta-model expanded in the previous section.
 
The front end user interface was implemented using a combination of HTML with Javascript and CSS, the principal framework used being Angular JS. Communication with the client was carried out using a REST controller, enabling a variety of clients potentially to link up with the Model Catalogue.

GORM was used as a persistence mechanism, with a MySQL relational database as storage, although different GORM adapters made it possible to attach NO SQL datastores such as Neo4J and MongoDB. The full architectural stack is shown in figure\ref{fig:ApplicationArchitectMDR}

\begin{figure}[here]
	\includegraphics[width=0.48\textwidth,natwidth=610,natheight=642]{ApplicationArchitect1}
	\caption{Overview Architecture} 
	\label{fig:ApplicationArchitectMDR}
\end{figure}

The Grails/GORM framework enabled the Ecore model to generate the basic Grails Domain model, and from that a Groovy DSL was built, using the Builder pattern, to handle transformations internally between different representational languages such as XML and Excel. A series of importers was built for data input from Excel, CSV, and various XML variants. Most XML structures are handled by transforming the XML from its native structure to our internal DSL-based XML structure. 

The internal domain model used a basic \emph{Catalogue Element} which was able to link elements via the \emph{relationship} and \emph{relationshipType} classes. The core language model discussed earlier has been enhanced to by allowing user-defined relationships to be added to the core model. Any catalogue element is able to be related to any other catalogue element through a relationship class, this relationship is constrained by the relationshipType object which can prevent different catalogue element types being related, so that a Model cannot directly be related to a say a Datatype Enumeration. Relationship types can be added to the Model dynamically, so that even though the relationship between a Model and Datatype enumeration is prevented initially, a new type could be introduced by an administrator or super user to add in that relationship. The EMF-based tools to automatically generate the whole Models Catalogue code-base using Groovy/Grails/Angular were not available at the start of the project, and although work has been undertaken to build such a toolkit it not yet complete.  

The following subsections describe the basic use cases for the Model Catalogue, and how these use cases were implemented. 
\subsection{Listing of DataModels}
The key use case required in both projects was the ability to catalogue a set of data elements and data classes so that different schema could be compared and curated. Listing is currently carried out using a REST interface which is queried using an Angular client. Figure \ref{fig:treeviewOfDataModel}
\begin{figure}[here]
	\includegraphics[width=0.5\textwidth,natwidth=610,natheight=642]{DataModelTreeView}
	\caption{Screen Shot of DataModel Listing} 
	\label{fig:treeviewOfDataModel}	
\end{figure}

\subsection{selection of data elements for form generation}
For many clinical users one of the key requirement was the ability to generate forms for clinical research which were generated form a single authorative source, and the ability to take data elements, manage them to build a form and then output either a form or an XML representation for use in another system, such as OpenClinica was key to the work carried out. Figure \ref{fig:treeviewOfDataModel} shows the interface for selecting and managing users selections of data elements and data classes.
\begin{figure}[here]
	\includegraphics[width=0.5\textwidth,natwidth=610,natheight=642]{DataElementSelection}
	\caption{Screen Shot of Data Element Selection} 
	\label{fig:treeviewOfDataModel}	
\end{figure}


\subsection{relationships between 2 DataModels}
Very often different research groups will arrive at slightly different models for the same or very similar diseased, so another use case for the Models Catalogue was the ability to compare different Data Models, Data Classes and individual Data Elements. Figure \ref{fig:dataClassComparison}
\begin{figure}[here]
	\includegraphics[width=0.5\textwidth,natwidth=610,natheight=642]{ComparisonOfDataClasses}
	\caption{Screen Shot of DataClass Comparison} 
	\label{fig:dataClassComparison}	
\end{figure}







%%-------------------------------
%%Drop next section
%%------------------------------
%%Some of the main relationships that are currently modelled in the \emph{Models Catalogue} are as follows:
 %%
%%\begin{center}
%%	\begin{tabular}{ p{1.5cm}  p{1.5cm}  p{1.5cm}   }  % centered columns (5 columns)
%%		Source & Relationship & Destination  \\
%%		Model & containment & DataElement   \\
%%	    Model & containment & Class    \\
%%	    Model & hierarchical & Model  \\
%%	    DataElement & supersession & DataElement  \\          
%%	\end{tabular}
%%\end{center}
%%
%%The Core architecture can seen in figure\ref{fig:ApplicationArchitectMDR} 
%%
%%\begin{figure}[here]
%%	\includegraphics[width=0.5\textwidth,natwidth=610,natheight=642]{System2}
%%	\caption{Overview Architecture} 
%%	\label{fig:System2MDR}
%%\end{figure}
%% 
\subsection{Modelling Overview}





 
 
\section{Experience}

\subsection{Re-use of data from clinical information systems}

The UK National Institute of Health Research (NIHR) is funding an
\pounds 11m programme of work across five large university-hospital
partnerships: at Oxford, Cambridge, Imperial College London,
University College London, and Guy's and St.~Thomas'.  The aim of the
programme is to create the infrastructure needed to support data
re-use and translational research across these five institutions.

The programme, the NIHR Health Informatics Collaborative (HIC), was
initiated in 2013, with a focus upon five therapeutic areas: acute
coronary syndromes, renal transplantation, ovarian cancer, hepatitis,
and intensive care.  The scope was increased in 2015 to include other
cancers---breast, colorectal, lung, and prostate---and other
infectious diseases, including tuberculosis.

The key component of the infrastructure consists in repositories of
patient data within each of the five institutions.  The intention is
that these repositories should hold a core set of data for each
therapeutic area, populated automatically from clinical systems,
together with detailed documentation on the provenance and
interpretation of each data point.  

Researchers can use the documentation to determine the availability
and suitability of data for a particular study.  They can use it also
to determine comparability across institutions: whether there are any
local differences in processes or equipment that would have a bearing
upon the combination and re-use of the corresponding data.  Once a
study is approved, the repositories act as a single source of data,
avoiding the need for data flows from individual clinical systems.

The development of the infrastructure required the development of a
`candidate data set' for each therapeutic area, as a core list of data
points collected in the course of routine care that would have value
also in translational research.  Each institution then set out to
determine which information systems, within their organisation, could
be used to populate each of the candidate data sets: this was termed
the `data exploration exercise'.

The results of the exercise informed further development of the data
sets, and data flows were established.  To demonstrate and evaluate
the new capability, `exemplar research studies' were initiated in each
therapeutic area, using data from all five institutions.  

Each institution had a different combination of existing systems, a
different approach to data integration, and a different strategy for
informatics development.  It was not feasible or appropriate to
develop a common `data repository' product for installation.  Instead,
a set of data models were distributed, and each institution worked to
implement these using their own messaging, business intelligence, or
data warehousing technologies. 

None of the institutions had the capability to provide documentation
on the provenance and interpretation of their data in any standard,
computable format; the model or metadata aspect of the infrastructure
was entirely new.  It was this that drove---and continues to
drive---the development of a comprehensive model catalogue
application. 

At the start of the project, teams of clinical researchers and leading
scientists were given the responsibility of creating the candidate
data sets for each therapeutic area.  They did this by exchanging
spreadsheets of data definitions in email.  This proved to be a slow
process, and face to face meetings were needed before any real
progress could be made.

It proved difficult to properly represent repeating sections of the
dataset---corresponding to investigations or interventions that may
happen more than once for the same patient.  Researchers resorted to
Visio diagrams to try to explain how observations fitted into clinical
pathways or workflows---and discovered that there were significant
differences between pathways for the same disease at different
institutions.  

In one therapeutic area, these differences had a profound effect upon
the interpretation of certain observations, and the candidate dataset
was extended to include additional information on the pathway.  Due to
the complexity of the pathways involved, this was a time-consuming and
error-prone process.  Furthermore, the spreadsheets quickly became
inconsistent with the Visio diagrams.

The candidate datasets were distributed to the informatics teams at
the five institutions in the form of XML schemas.  At first, these
were created from scratch, rather than being generated.  There were
many requests for changes to the schemas; these proved difficult to
track and coordinate.

The exploration exercise was reported by adding columns to the
distributed versions of the candidate dataset spreadsheets, listing
the information systems containing the data points in question, or
suggested alternatives where there were significant differences due to
local systems and processes.

This was despite the availability of an initial version of the model
catalogue.  Researchers and local informatics teams preferred to work
with spreadsheets, having little or no knowledge of modelling
languages such as UML and no automatic support for model creation and
maintenance.  It fell to the software engineering team at the
coordinating centre to record the datasets and variations in the
catalogue.

While it was disappointing to have the researchers still working in
spreadsheets, the ability to generate XML schemas from models, and to
manage relationships between data items in different models and
different versions, proved invaluable.  In the second phase of the
project, with more therapeutic areas being added, researchers are
starting to abandon the spreadsheet mode of working, and are instead
maintaining the datasets as data models, in the catalogue.

\subsection{Coordination of clinical data acquisition}

The UK Department of Health, through the NIHR and the National Health
Service (NHS), is providing funding for the whole genome sequencing of
blood and tissue samples from patients with cancer, rare disorders,
and infectious disease.  A network of regional centres is being
established to collect samples and data, and to provide access to
genomic medicine across the whole of the country.  The funding
committed to date is approximately \pounds 300m.

The results of the whole genome sequencing will be linked to detailed
information on each participant: clinical and laboratory information
drawn from health records, ontological statements regarding abnormal
features or conditions, and additional information obtained from the
participant or their representatives.  The information required will
depend upon the nature of the disease that the patient is suffering
from.  For example, information on breast density is required in the
case of breast cancer, but not for other diseases.

131 different diseases have been included in the sequencing programme
thus far.  Each disease corresponds to a different combination of
clinical and laboratory data points, a different set of ontological
statements, and a different set of questions for the participant.
There are, however, significant overlaps between diseases: for
example, many different rare diseases will require the same
information on kidney or heart function.   

The modelling task is at least an order of magnitude greater than that
required for the NIHR HIC, and yet candidate datasets have already
been created for more than half of the diseases included.  This is due
partly to the availability of the model catalogue application from the
start of the project, and partly to the availability, within the
catalogue, of the full complement of HIC-defined data models and
related data sets---including the national NHS data dictionary and the
national cancer reporting datasets.

The informatics infrastructure required for the genomic medicine
programme needs to support frequent extensions or revisions to the
models proposed.  As the analysis of the genomic data proceeds, new
questions will be asked, and new information will be required from the
health records.  

Two routes are available for the provision of data from the network of
contributing centres: direct data entry into electronic case report
forms, in a on-line clinical trials management system; and electronic
submission of data in XML format.  The intended interpretation of the
data required is explained in a regularly-updated set of data
manuals.   

It is important that the forms used for direct data entry, the schemas
used for XML submission, and the data manuals are properly
synchronised.  An initial approach to this, in which a single model
was used as the basis for the generation of all three kinds of
artefact, proved inconvenient in practice.  Although the same data
points were to be collected in each case, the distribution of these
data points across classes and sections was different.

Accordingly, the model catalogue is used to store three different
models for each dataset: one for the generation of the forms, another
for the generation of the XML schemas, and one for the generation of
the data manual.  These models are semantically-linked.  If one is
updated, then the fact that the others may now be inconsistent will be
flagged to the user.  

The same linkage is made with regard to existing reporting datasets
and clinical audits.  To avoid duplication of effort, the reporting
datasets for the genomic medicine programme have been aligned with
these activities.  The existing datasets have been modelled, and
updates to them will be tracked in the catalogue: again, potential
inconsistencies can be flagged.

The degree of data definition re-use facilitated by the model is
promising.  Table~\ref{table:reuse} shows the percentage of data
definitions in the core cancer models developed for the genomic
medicine programme that could be drawn from the two main, national
cancer reporting datasets---the Cancer Outcomes and Services Dataset
(COSD) and the Systemic Anti-Cancer Therapy dataset (SACT)---or from
the standard NHS data dictionary.

\begin{table}[h]
  \caption{Data definition re-use for cancer models}
  \label{table:reuse}
  \begin{tabular}{lrr}
    \hline
    Source &  Data Elements & \%  \\ 
    \hline
    Cancer Outcomes and Services Reporting  & 60 & 28.45 \% \\
    Systemic Anti-Cancer Therapy Reporting & 2 & 0.95 \% \\
    NHS Data Dictionary & 12 & 5.7 \% \\
    New Definitions & 137 & 64.9 \% \\
    \hline
    TOTAL & 211
  \end{tabular}
\end{table}

The degree of data definition re-use for rare disease is slightly
lower than this, although still promising.  There is no existing
national reporting or registry activity for many of the rare disease
areas, and the datasets used by individual research programmes are not
readily available.   


\input{5Conclusion.tex}
 


\newpage

\bibliographystyle{plain}

\bibliography{ASEPub}


\end{document}  

%%% Local Variables:
%%% mode: latex
%%% TeX-master: "ASE2015MDR"
%%% End:
