\section{Data Models and Metadata}

\subsection{Data models}

A dataset definition for clinical research will consist of a number of
different parts, each of which declares a set of related data items.
Typically, this will be a set of data items that would be collected
together: in the completion of a `case report form' in a study, or in
the report of a clinical event.  These parts may be `repeating', in
that there may be more than one form of that type completed, or more
than one event of that type reported upon.  

A data item declaration should explain not only the name under which
values are to be stored, but also the type of those values.  If the
type is numeric, then the unit of measurement should be given.  If the
type is an enumeration, then the intended interpretation of each value
should be explained.  Finally, the parts of the dataset may be
connected or related to one another, and these relationships may have
constrained multiplicities.

It should be clear that a dataset definition can be represented as a
class diagram or object model.  Data items can be introduced as
attributes, parts of the model as classes, and relationships between
classes as associations---complete with multiplicities.  Data types
and enumerations can be used to support attribute declarations, and
hence data item definitions; constraints can be used to express range
restrictions and intended relationships between data items.

Class diagrams can be used also to provide precise definitions for
data sources: existing clinical systems or research databases.  These
systems may be implemented using a combination of technologies but,
assuming that the value of the data justifies the effort required,
precise object models can be created to describe the interfaces that
they might present for data re-use. 

\subsection{Models as metadata}

Having constructed precise object models to describe datasets and data
sources, we can use them as a basis for automated software and data
engineering.  This involves the use of models as metadata in three
respects: for software artefacts that are generated or configured; for
data that these artefacts store or process; and for other models,
capturing relationships between data points declared in different
models, and hence the data stored or processed against them.

For the first of these, we need only maintain the relationship between
the implementation and the model that was used to generate or
configure it.  If the model is stored in a repository or \emph{model
  catalogue} and published, then we have only to create a link between
the implementation and this published instance.  This link may be
created automatically in the case where the model is generated,
partially or completely, from an existing implementation: for example,
from the schema of a relational database.

For the second, a link must be created between each data point, or
collection of data points, and the corresponding declaration in the
data model.  The model catalogue then requires some support for the
language in which the data model is expressed, sufficient to support
the creation and manipulation of links to specific components: in
particular, to specific classes and attributes.  If the data is
acquired or processed using a software implementation for which a data
model has already been registered, then that model may provide a basis
for the automatic creation of a link to the corresponding declaration.

For the third, we need the same ability to create and manipulate links
between components of different models.  Instead of links between
model and implementation, however, we have links representing
different kinds of relationships: provenance and re-use of model
components; versioning of models; and, most importantly,
\emph{semantic relationships} between attributes or
classes---assertions that one attribute or class has the same meaning
or interpretation as another.  This is precisely the information
needed to support automation in data discovery, integration, and
re-use.

\subsection{Data Model Language}

The core features of our modelling language are familiar: classes,
associations, attributes, constraints, enumerations, data types.
However, our interpretation of these features is narrower than that
set out in, for example, the specifications of the Unified Modeling
Language (UML)~\cite{UML} or Ecore~\cite{ECORE}.  For this reason, and
because of the need to refer explicitly to models and capture semantic
relationships between models, classes, and attributes, we define a
domain specific language for data models.

We extend this language to include additional modelling features
pertaining to certain kinds of implementation artefacts, such as
forms, messages, databases, and documentation.  This is necessary only
when the feature in question may have a bearing upon the data
semantics, and may thus serve as a source or target for a link
describing a property or relationship.  

For example, a section of a form, corresponding to one or more classes
of data, might comprise information about allergies.  We may wish to
link this section to a term in an ontology recording this fact.  This
makes it easier for someone interested in models of allergy
information to locate and possibly re-use the definitions; it also
adds context to any semantic links to data classes and attributes
associated with that section.

The generated or configured part of an implementation may include
features not described by the data model.  This will be the case
wherever a model-driven or generative approach has been adopted in the
development of an application.  In this case, the artefact generated
from or represented by the data model is itself a model, in a
different modelling language.  The use of a data model, in our domain
specific language, as metadata for this other model should come as no
surprise: it is no different from the use of a model as metadata for
program source code. 
