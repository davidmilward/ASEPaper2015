\section{The Models Catalogue Language and Architecture}
 
\subsection{Models Catalogue}
A software \emph{model} is an abstraction of a software program, which will be instantiated and run using \emph{real data}, software modelling is becoming important for a number of reasons, not least that it enables developers the opportunity to \emph{re-use} code. The Unified Modelling Language UML \cite{UML} or Ecore \cite{ECORE} ( a subset of UML) are in widespread use for defining and representing software \emph{models}. UML and Ecore are defined at a level which is \emph{more abstract} than the model itself, and one which we term the \emph{meta-modelling} layer. 

The data we have been dealing with in these projects is mostly in the form of textual forms, sometimes in word or PDF format, sometimes in a more specialised form of XML or Excel. The data therefore does have a clear structure, although the clinician running the trial very often has little visibility, knowledge or appreciation of how the data is stored, and who else might in future want to access or manipulate that data.  Although the healthcare providers have work-flows which the patients and doctors follow in the treatment cycle, these are not being considered in this paper.

Sometimes there are standards available governing aspects of the data in question, for instance reporting of a particular clinical trial may need to be in XML which conforms to a particular XSD, however that doesn't mean that the data has been collected with that particular standard in mind.  For the most part clinicians are interested in fairly static data sets, a patient record, the results of a treatment, a clinical trial form, the dynamic aspects are less important, at any rate in terms of the computational representation. What is important is the meaning of the data once it is captured, and in particular how it relates to other data in the same field, for instance is \emph{tumour weight} in one experiment the same as \emph{tumour weight} in another. These two terms may be used in different contexts, for instance prostate cancer and liver cancer, and they may be represented by different values domain in different hospital trusts, for instance one may use grams the other kilograms, the values may be comparable but an adjustment will need to be made to make that comparison.

Thus there are two things we need to identify and reference in building the toolkit, firstly does this \emph{concept} \textbf{mean} the same as that \emph{concept}, and secondly does this \emph{concept} have the same \emph{representation} as that concept. The way in which we can match up concepts in software engineering is by associated them with classes of software objects, and then defining methods which can reference them. 

\subsection{ISO11179}
ISO11179 is the international standard relating to metadata and in particular metadata registries. It forms the basis of the design of the models catalogue toolkit.

\begin{figure}[here]
	\includegraphics[width=0.48\textwidth,natwidth=610,natheight=642]{BasicISO}
	\caption{Core model for ISO11179 Metadata Registry} 
	\label{fig:basicMDR}
\end{figure}

The ISO11179 standard uses the notion of a \emph{data element concept}, \emph{a data element}, \emph{a value domain}, and a \emph{conceptual domain}. The standard currently confines itself to the detailed level of concepts and data elements and has no notion of collections of data elements or data element concepts, but instead attaches two attributes: an \emph{object class} and a \emph{property} to each data element concept and these attributes allow the data element concept's to be aggregated or classified. This core model of the ISO11179 is illustrated in UML notation in figure \ref{fig:basicMDR}. The data element concept and conceptual domain entities belong to the \emph{conceptual level} n whilst the common data element and value domain both belong to the \emph{representational level}.


For example an Integer data-type in a programming language may be used to represent inches in a measurement program, it may also be used to count vehicles in a logistics application.  A data element is said to be comprised of a data element concept(DEC) which is its meaning and a value domain(VD) which is its representation.


\begin{table}[h]
	\begin{tabular}{ p{1.8cm} p{2.8cm}  p{3.0cm}  }  % centered columns (2 columns)
		\hline
		Entity & ISO Definition & ISO11179 Implementation Guidelines  \\ 
		\hline
		Data Element Concept(DEC) & An idea that can be represented in the form of a data element, described independently of any particular representation. & A concept that can be represented in the form of a Data Element, described independently of any particular representation.\\
		Common Data Element(CDE) & A unit of data for which the definition, identification, representation, and permissible values are specified by means of a set of attributes. & A unit of data for which the definition, identification, representation and Permissible Values are specified by means of a set of attributes. \\
		Value Domain (VD) & The description of a value meaning. & A description of a Value Meaning. \\
		Conceptual Domain (CD) & A set of valid value meanings, which may be enumerated or expressed via a description.& A set of valid Value Meanings.\\
		\hline
	\end{tabular}
\end{table}

\vspace{5.mm}



 

Consider the definition of a \emph{Conceptual Domain}:
\begin{quote}
	A conceptual domain is a set of value meanings. The intention of a conceptual domain is to detail the model's value meanings. Many value domains may be in the extension of the same conceptual domain, but a value domain is associated with one conceptual domain. Conceptual domains may have relationships with other conceptual domains, so it is possible to create a concept system of conceptual domains. Value domains may have relationships with other value domains, which provide the framework to capture the structure of sets of related value domains and their associated concepts.	
\end{quote}
Conceptual domains comprise sets of value domains, and enable value domains themselves to be ``re-used'' between difference Data Elements. 






\subsection{A Metadata Language and Abstract Architecture }

The ISO11179 specification illustrates different aspects of the standard using UML class diagrams, we have used these to inform the development of a very simple domain specific language, which we are calling \textbf{Elm DSL} based on the standard. The main differences are that we have added in containers to handle data element collections, calling these \emph{Classes} and to handle collections of these \emph{Classes} which we have called \emph{Models}.

A simplified overview model, without attributes and methods, showing the Ecore model for the Elm DSL is shown in Figure \ref{fig:mcSimplifiedOverview}.

\begin{figure}[here]
	\includegraphics[width=0.5\textwidth,natwidth=610,natheight=642]{ELM_EcoreDiagram}
	\caption{Overview of LEM in Ecore} 
	\label{fig:mcSimplifiedOverview}
\end{figure}

\subsubsection{Model}
A model is a grouping or containment entity which groups a set of \emph{Classes} together. Models can be thought of as datasets, or even database schemas, very often in the medical domain they are defined either by XML Schema definition files, or by equivalent schemas written in Excel. 
Models are collections of either \emph{Data Elements} or \emph{Classes}. This aspect is captured in the \emph{Emfatic} representation by the \emph{AbstractElement} which can be either a  \emph{Data Element} or a \emph{Class}  There is no real notion of composition or multiplicity, a instance of a Model can contain an instance of a Data Element or not as required by the instance.  Models are named, have a description and have a version identity.
\subsubsection{Class}
A Class is a grouping or collection of \emph{attributes} which can be data elements or classes, the attributes are currently \emph{mandatory}, so that class with 5 attributes must have those 5 attributes instantiated in an instance for it to be considered of that class. Classes represent \emph{Concepts}, (***This is not true present****)and can be \emph{Generalized} into a hierarchy(**but we might get it implemented before the paper comes out**). The idea of a \emph{Data Element Concept} was omitted from the initial prototype, although it has been added to later versions of the language.
\subsection{Data Elements} 
Data Elements can also represent \emph{Concepts} and are by their nature \emph{atomic}.  Each data element is related to a value domain on a one-to-one basis, and the relationship is a two-way relationship.
\subsection{Value Domain}
A Value Domain is the domain in which the data element is represented, it can consist of one or more \emph{ValueSpecs}.
\subsection{ValueSpec}
A \emph{ValueSpec} can be a simple datatype, an enumeration of datatypes, or a rule - such as a regular expression - which defines the way in which a series of characters is formed into a string attribute. 




\lstinputlisting[label=elm,caption=Language for Enterprise Modelling]{ASEFigs/ElmmDSL.xtext}


Data Dictionaries can be used to store details of a database record structure or even an application data structure on a local 'per-application' basis, a metadata registry provides a similar capability but on a system or organisation-wide basis. It also provides features that are commonly included in a \emph{Thesaurus}, a \emph{Taxonomy}, and an \emph{Ontology}. These features include the ability to classify terms in relation to one another, record relationships such as synonyms, and classify hierarchical relationships. Ontologies have proved effective in matching what we have defined as the first problem, that is how domain concepts can be matched, however they are quite unwieldy to use when tackling the second problem of how to match and manage different representations. The ISO11179 standard came about in an attempt to define what was need to store and manage \emph{metadata}, it has been revised twice and is currently in its third edition. We took the standard as an initial reference, built a prototype and then discovered that for our purposes in providing a means to control the datasets used in clinical trials that further work was needed in order to provide an effective toolkit, and in order to differentiate our toolkit from the standard \emph{metadata registry} we are calling it a \textbf{models catalogue}.

The remainder of this sections gives a brief review of the fundamental concepts in ISO11179, the second subsection describes how we modified the ideas in ISO11179 to develop our models catalogue, and the last subsection gives and overview of the meta-modelling domain specific language that was developed as a by-product. 

 




