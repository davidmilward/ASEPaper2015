\section{Implementation}

Groovy was chosen as a language for implementation for two reasons, firstly
it has a very efficient web framework called Grails built on the Spring framework,
which is not only proven to be very robust and scalable, but is also
very easy to implement and so enables quick development cycles. Previous
implementations using Java/Spring and Java/Roo have proved very timeconsuming
to experiment with, Grails has proven to be very flexible. The
second reason was that Grails offers both functional capability, and dynamic
meta-programming capability. This means that DSL’s can be easily built on
this framework, something that hasn’t been used yet, but will be needed when
the automatic forms generation capability is added. At this stage I have built 2
forms generating DSLs, although must more work will be required to get one
that can implement automatic forms generation.
 
The MDR implementation discussed here has been implemented using the
Groovy programming language, and the Grails Framework and is based on
the concepts put forward in the ISO11179 standard. There are a few changes
to basic model in that the idea of a Data Element Concept has been merged
with the idea of aModel. Data Elements have been retained, and they are related
to both aModel and a Value Domain. Models can contain other models,
and are essentially the basic building block of the data definitions, however
at their lowest resolution a model will be composed of a Data Element and at
least one Value Domain.
 