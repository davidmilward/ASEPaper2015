
\section{Conclusion }

\section{Lessons learned}

\begin{itemize}
\item metadata registries are not enough - you need models - (we
  should include examples  of how data elements are contextualised -
  Keith slides) - 
\item these models should be linked to the implementation, for
  otherwise it is too expensive to maintain them
\item the models should be readable as data manuals - literate
  modelling - it simply doesn't work to have to click on attributes to
  understand their meaning and derivation 
\item this readability applies to the development interface, this is
  the interface that you want to use (can't do the turn time) - and
  you need to be able to pair program with domain experts
\item you want inheritance within models, and include/import across
  models (start thinking of the models as code, although this is about
  managing information in general - managing declarations - it is
  coincidental that we are generating artefacts from them)
\item you may want different types of model for different generated
  artefacts---depending upon what, exactly, is in the generator code
  as parameters and additional information---it may be the same set of
  data points overall, but you might well have a different refactoring
  in Fowler terms
\item version and store the generation code (!)
\item tag the generated artefacts with a link back to the model
  catalogue (XML schemas in HIC!)
\item you will want more models than you think: for example, consider
  the model of a clinical test; now think about how it is dropped into
  a pathway; you need the pathway model to tell you what the dataset
  means; it has added further context to the data element definitions
  (you could flatten this, but that's bad)
\item automate aspects of model management---versioning and dealing
  with multiple models
  \begin{itemize}
  \item automatically create (and propose) links, including
    classifications
  \item have links for new version of (sequential), refactoring of
    (parallel), derived from (data concepts across models), same as
    (strong assertion)
  \item use links in model maintenance - note that the definition that
    this was derived from has changed
  \item have publication cycle---a published model is for life
  \end{itemize}
\item general lesson: if you want to manage data semantics, you need a
  compositional approach---none of this central coordination of a
  single data dictionary, none of this single hierarchy of data
  elements, none of this element by element description (all the
  context in the text of the element definition?  doesn't work!)
\item general lesson: if you are going to manage data at scale, you
  need data model driven approach
\item general lesson: if you have a model driven approach (data model
  or otherwise) then your management of models has all the same
  challenges as management of source code (you need an IDE) - really,
  if you are using models as programs, then you need to support them
  as programs
\end{itemize}

\subsection{Future work}



