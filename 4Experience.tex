\section{Experience}

\subsection{Re-use of data from clinical information systems}

The UK National Institute of Health Research (NIHR) is funding an
\pounds 11m programme of work across five large university-hospital
partnerships: at Oxford, Cambridge, Imperial College London,
University College London, and Guy's and St.~Thomas'.  The aim of the
programme is to create the infrastructure needed to support data
re-use and translational research across these five institutions.

The programme, the NIHR Health Informatics Collaborative (HIC), was
initiated in 2013, with a focus upon five therapeutic areas: acute
coronary syndromes, renal transplantation, ovarian cancer, hepatitis,
and intensive care.  The scope was increased in 2015 to include other
cancers---breast, colorectal, lung, and prostate---and other
infectious diseases, including tuberculosis.

The key component of the infrastructure consists in repositories of
patient data within each of the five institutions.  The intention is
that these repositories should hold a core set of data for each
therapeutic area, populated automatically from clinical systems,
together with detailed documentation on the provenance and
interpretation of each data point.  

Researchers can use the documentation to determine the availability
and suitability of data for a particular study.  They can use it also
to determine comparability across institutions: whether there are any
local differences in processes or equipment that would have a bearing
upon the combination and re-use of the corresponding data.  Once a
study is approved, the repositories act as a single source of data,
avoiding the need for data flows from individual clinical systems.

The development of the infrastructure required the development of a
`candidate data set' for each therapeutic area, as a core list of data
points collected in the course of routine care that would have value
also in translational research.  Each institution then set out to
determine which information systems, within their organisation, could
be used to populate each of the candidate data sets: this was termed
the `data exploration exercise'.

The results of the exercise informed further development of the data
sets, and data flows were established.  To demonstrate and evaluate
the new capability, `exemplar research studies' were initiated in each
therapeutic area, using data from all five institutions.  

Each institution had a different combination of existing systems, a
different approach to data integration, and a different strategy for
informatics development.  It was not feasible or appropriate to
develop a common `data repository' product for installation.  Instead,
a set of data models were distributed, and each institution worked to
implement these using their own messaging, business intelligence, or
data warehousing technologies. 

None of the institutions had the capability to provide documentation
on the provenance and interpretation of their data in any standard,
computable format; the model or metadata aspect of the infrastructure
was entirely new.  It was this that drove---and continues to
drive---the development of a comprehensive model catalogue
application. 

At the start of the project, teams of clinical researchers and leading
scientists were given the responsibility of creating the candidate
data sets for each therapeutic area.  They did this by exchanging
spreadsheets of data definitions in email.  This proved to be a slow
process, and face to face meetings were needed before any real
progress could be made.

It proved difficult to properly represent repeating sections of the
dataset---corresponding to investigations or interventions that may
happen more than once for the same patient.  Researchers resorted to
Visio diagrams to try to explain how observations fitted into clinical
pathways or workflows---and discovered that there were significant
differences between pathways for the same disease at different
institutions.  

In one therapeutic area, these differences had a profound effect upon
the interpretation of certain observations, and the candidate dataset
was extended to include additional information on the pathway.  Due to
the complexity of the pathways involved, this was a time-consuming and
error-prone process.  Furthermore, the spreadsheets quickly became
inconsistent with the Visio diagrams.

The candidate datasets were distributed to the informatics teams at
the five institutions in the form of XML schemas.  At first, these
were created from scratch, rather than being generated.  There were
many requests for changes to the schemas; these proved difficult to
track and coordinate.

The exploration exercise was reported by adding columns to the
distributed versions of the candidate dataset spreadsheets, listing
the information systems containing the data points in question, or
suggested alternatives where there were significant differences due to
local systems and processes.

This was despite the availability of an initial version of the model
catalogue.  Researchers and local informatics teams preferred to work
with spreadsheets, having little or no knowledge of modelling
languages such as UML and no automatic support for model creation and
maintenance.  It fell to the software engineering team at the
coordinating centre to record the datasets and variations in the
catalogue.

While it was disappointing to have the researchers still working in
spreadsheets, the ability to generate XML schemas from models, and to
manage relationships between data items in different models and
different versions, proved invaluable.  In the second phase of the
project, with more therapeutic areas being added, researchers are
starting to abandon the spreadsheet mode of working, and are instead
maintaining the datasets as data models, in the catalogue.

\subsection{Coordination of clinical data acquisition}

The UK Department of Health, through the NIHR and the National Health
Service (NHS), is providing funding for the whole genome sequencing of
blood and tissue samples from patients with cancer, rare disorders,
and infectious disease.  A network of regional centres is being
established to collect samples and data, and to provide access to
genomic medicine across the whole of the country.  The funding
committed to date is approximately \pounds 300m.

The results of the whole genome sequencing will be linked to detailed
information on each participant: clinical and laboratory information
drawn from health records, ontological statements regarding abnormal
features or conditions, and additional information obtained from the
participant or their representatives.  The information required will
depend upon the nature of the disease that the patient is suffering
from.  For example, information on breast density is required in the
case of breast cancer, but not for other diseases.

131 different diseases have been included in the sequencing programme
thus far.  Each disease corresponds to a different combination of
clinical and laboratory data points, a different set of ontological
statements, and a different set of questions for the participant.
There are, however, significant overlaps between diseases: for
example, many different rare diseases will require the same
information on kidney or heart function.   

The modelling task is at least an order of magnitude greater than that
required for the NIHR HIC, and yet candidate datasets have already
been created for more than half of the diseases included.  This is due
partly to the availability of the model catalogue application from the
start of the project, and partly to the availability, within the
catalogue, of the full complement of HIC-defined data models and
related data sets---including the national NHS data dictionary and the
national cancer reporting datasets.

The informatics infrastructure required for the genomic medicine
programme needs to support frequent extensions or revisions to the
models proposed.  As the analysis of the genomic data proceeds, new
questions will be asked, and new information will be required from the
health records.  

Two routes are available for the provision of data from the network of
contributing centres: direct data entry into electronic case report
forms, in a on-line clinical trials management system; and electronic
submission of data in XML format.  The intended interpretation of the
data required is explained in a regularly-updated set of data
manuals.   

It is important that the forms used for direct data entry, the schemas
used for XML submission, and the data manuals are properly
synchronised.  An initial approach to this, in which a single model
was used as the basis for the generation of all three kinds of
artefact, proved inconvenient in practice.  Although the same data
points were to be collected in each case, the distribution of these
data points across classes and sections was different.

Accordingly, the model catalogue is used to store three different
models for each dataset: one for the generation of the forms, another
for the generation of the XML schemas, and one for the generation of
the data manual.  These models are semantically-linked.  If one is
updated, then the fact that the others may now be inconsistent will be
flagged to the user.  

The same linkage is made with regard to existing reporting datasets
and clinical audits.  To avoid duplication of effort, the reporting
datasets for the genomic medicine programme have been aligned with
these activities.  The existing datasets have been modelled, and
updates to them will be tracked in the catalogue: again, potential
inconsistencies can be flagged.

The degree of data definition re-use facilitated by the model is
promising.  Table~\ref{table:reuse} shows the percentage of data
definitions in the core cancer models developed for the genomic
medicine programme that could be drawn from the two main, national
cancer reporting datasets---the Cancer Outcomes and Services Dataset
(COSD) and the Systemic Anti-Cancer Therapy dataset (SACT)---or from
the standard NHS data dictionary.

\begin{table}[h]
  \caption{Data definition re-use for cancer models}
  \label{table:reuse}
  \begin{tabular}{lrr}
    \hline
    Source &  Data Elements & \%  \\ 
    \hline
    Cancer Outcomes and Services Reporting  & 60 & 28.45 \% \\
    Systemic Anti-Cancer Therapy Reporting & 2 & 0.95 \% \\
    NHS Data Dictionary & 12 & 5.7 \% \\
    New Definitions & 137 & 64.9 \% \\
    \hline
    TOTAL & 211
  \end{tabular}
\end{table}

The degree of data definition re-use for rare disease is slightly
lower than this, although still promising.  There is no existing
national reporting or registry activity for many of the rare disease
areas, and the datasets used by individual research programmes are not
readily available.   

